\section{Transforming Crystal Field Parameters  by Rotation of the Coordinate %%@
System}\label{rotateBlm}

There is a special program {\prg cf2mcphas} for rotating crystal field parameters
(Stevens Notation) 
given in the orientation $abc||$xyz to the {\prg McPhase} notation $abc||$yzx. The 
original crystal field parameters have to be given in a file of the above format of a %%@
single
ion parameter file. Output is written to stdout. Note, there is also a program {\prg
mcphas2cf} for transforming parameters back. For more general rotations of 
crystal field parameters you can use the program {\prg rotateBlm}(see %%@
chapter~\ref{addprog}).

As an example we consider a tetragonal system, where the crystal field parameters
are usually given such that  $abc||$xyz - we denote these parameters as
 $B_l^m$. For tetragonal symmetry only $B_2^0, B_4^0, B_4^4, B_6^0$ and $B_6^6$ are
 not zero. In order to use these parameters in {\prg mocule cfield}, they 
 have to be transformed to a coordinate system $xyz$ with $abc||yzx$. Denoting the new
 crystal field parameters which are to be used as $Blm$ the transformation
 is given by 
 
\begin{eqnarray}
 B20&=&-\frac{1}{2} B_2^0  \nonumber \\  
 B22&=&+\frac{3}{2} B_2^0  \nonumber \\
 B40&=&+\frac{3}{8} B_4^0+\frac{1}{8} B_4^4     \nonumber \\
 B42&=&-\frac{5}{2} B_4^0+\frac{1}{2} B_4^4    \nonumber \\
 B44&=&\frac{35}{8} B_4^0+\frac{1}{8} B_4^4  \nonumber \\
 B60&=&-\frac{5}{16} B_6^0-\frac{1}{16} B_6^4  \nonumber \\
 B62&=&+\frac{105}{32} B_6^0+\frac{5}{32} B_6^4  \nonumber \\
 B64&=&-\frac{63}{16} B_6^0+\frac{13}{16} B_6^4  \nonumber \\
 B66&=&\frac{231}{32} B_6^0+\frac{11}{32} B_6^4  \nonumber \\
\end{eqnarray}

As a second example we consider an orthorhombic system, where the crystal field %%@
parameters
are given such that  $abc||$xyz - we denote these parameters as
 $B_l^m$. For orthorhombic symmetry only $B_2^0,B_2^2, B_4^0,B_4^2, B_4^4, B_6^0, B_6^2, %%@
B_6^4$ and $B_6^6$ are
 not zero. In order to use these parameters in {\prg mocule cfield}, they 
 have to be transformed to a coordinate system $xyz$ with $abc||yzx$. Denoting the new
 crystal field parameters which are to be used as $Blm$ the transformation
 is given by 
 
\begin{eqnarray}
 B20&=&-\frac{1}{2} B_2^0 -\frac{1}{2} B_2^2  \nonumber \\  
 B22&=&+\frac{3}{2} B_2^0 -\frac{1}{2} B_2^2  \nonumber \\
 B40&=&+\frac{3}{8} B_4^0 +\frac{1}{8} B_4^2 +\frac{1}{8} B_4^4     \nonumber \\
 B42&=&-\frac{5}{2} B_4^0 -\frac{1}{2} B_4^2 +\frac{1}{2} B_4^4    \nonumber \\
 B44&=&\frac{35}{8} B_4^0 -\frac{7}{8} B_4^2 +\frac{1}{8} B_4^4  \nonumber \\
 B60&=&-\frac{5}{16} B_6^0-\frac{1}{16} B_6^2-\frac{1}{16} B_6^4-\frac{1}{16} B_6^6  %%@
\nonumber \\
 B62&=&+\frac{105}{32} B_6^0+\frac{5}{32} B_6^4+...  \nonumber \\
 B64&=&-\frac{63}{16} B_6^0+\frac{13}{16} B_6^4+...  \nonumber \\
 B66&=&\frac{231}{32} B_6^0+\frac{11}{32} B_6^4+...  \nonumber \\
\end{eqnarray}

As a second example we consider an dhcp system with quasi cubic sites, where the crystal %%@
field parameters
are given such that  $abc||$xyz - we denote these parameters as
 $B_l^m$. Only $B_2^0,B_2^{-1},B_2^2, B_4^0,B_4^{-1},B_4^2,B_4^{-3}, B_4^4, %%@
B_6^0,B_6^{-1}, B_6^2,B_6^{-3}, B_6^4$ and $B_6^6$ are
 not zero. In order to use these parameters in {\prg mocule cfield}, they 
 have to be transformed to a coordinate system $xyz$ with $abc||yzx$. Denoting the new
 crystal field parameters which are to be used as $Blm$ the transformation
 is given by 
 
\begin{eqnarray}
 B21&=&-2B_2^{-1}  \nonumber \\  
 B20&=&-\frac{1}{2} B_2^0 -\frac{1}{2} B_2^2  \nonumber \\  
 B22&=&+\frac{3}{2} B_2^0 -\frac{1}{2} B_2^2  \nonumber \\
 B4-3&=&-\frac{7}{4} B_4^{-1} -\frac{3}{4} B_4^{-3}  \nonumber \\
 B4-1&=&+\frac{3}{4} B_4^{-1} -\frac{1}{4} B_4^{-3}  \nonumber \\
 B40&=&+\frac{3}{8} B_4^0 +\frac{1}{8} B_4^2 +\frac{1}{8} B_4^4     \nonumber \\
 B42&=&-\frac{5}{2} B_4^0 -\frac{1}{2} B_4^2 +\frac{1}{2} B_4^4    \nonumber \\
 B44&=&\frac{35}{8} B_4^0 -\frac{7}{8} B_4^2 +\frac{1}{8} B_4^4  \nonumber \\
 B60&=&-\frac{5}{16} B_6^0-\frac{1}{16} B_6^4+...  \nonumber \\
 B62&=&+\frac{105}{32} B_6^0+\frac{5}{32} B_6^4+...  \nonumber \\
 B64&=&-\frac{63}{16} B_6^0+\frac{13}{16} B_6^4+...  \nonumber \\
 B66&=&\frac{231}{32} B_6^0+\frac{11}{32} B_6^4+...  \nonumber \\
\end{eqnarray}

\end{description}

The general transformation matrices are given by

\subsubsection{Forward Rotation Matrices} \label{ap-cf2mcphMat}

These matrices rotate a set of crystal field parameters (or Stevens operators) by %%@
$\Theta$=90$^o$ and $\Phi$=90$^o$.

\begin{equation} \label{eq:S2}
\V{S}_2(\pi/2,\pi/2) = \left(
\begin{array}{ccccc}
 0 &    0 &    0 &  1/2 &    0 \\
-2 &    0 &    0 &    0 &    0 \\
 0 &    0 & -1/2 &    0 & -1/2 \\
 0 &    1 &    0 &    0 &    0 \\
 0 &    0 &  3/2 &    0 & -1/2 \\
\end{array} \right)
\end{equation}

%\begin{widetext}
\begin{equation} \label{eq:S4}
\V{S}_4(\pi/2,\pi/2) = \left(
\begin{array}{ccccccccc}
 0 &    0 &    0 &    0 &    0 &    0 &    0 & -1/8 &    0 \\
 1 &    0 & -7/2 &    0 &    0 &  7/8 &    0 &    0 &    0 \\
 0 &    0 &    0 &    0 &    0 &    0 &    0 & -1/4 &    0 \\
 1 &    0 &  1/2 &    0 &    0 & -1/4 &    0 &    0 &    0 \\
 0 &    0 &    0 &    0 &  3/8 &    0 &  1/8 &    0 &  1/8 \\
 0 & -2/5 &    0 & -3/4 &    0 &    0 &    0 &    0 &    0 \\
 0 &    0 &    0 &    0 & -5/2 &    0 & -1/2 &    0 &  1/2 \\
 0 & -3/4 &    0 &  7/4 &    0 &    0 &    0 &    0 &    0 \\
 0 &    0 &    0 &    0 & 35/8 &    0 & -7/8 &    0 &  1/8 \\
\end{array} \right)
\end{equation}

{\scriptsize
\begin{equation} \label{eq:S6}
\V{S}_6(\pi/2,\pi/2) = \left(
\begin{array}{ccccccccccccc}
   0 &    0 &    0 &      0 &     0 &      0 &      0 &     0 &      0 & -11/32 &     0 & %%@
1/32 &      0 \\
-3/8 &    0 & 11/4 &      0 & -33/8 &      0 &      0 & 33/32 &      0 &      0 &     0 &    %%@
0 &      0 \\
   0 &    0 &    0 &      0 &     0 &      0 &      0 &     0 &      0 &   -3/8 &     0 &  %%@
1/8 &      0 \\
-5/8 &    0 &  5/4 &      0 &   9/8 &      0 &      0 &  -3/8 &      0 &      0 &     0 &    %%@
0 &      0 \\
   0 &    0 &    0 &      0 &     0 &      0 &      0 &     0 &      0 &   9/32 &     0 & %%@
5/32 &      0 \\
-3/4 &    0 & -1/2 &      0 &  -1/4 &      0 &      0 &  5/32 &      0 &      0 &     0 &    %%@
0 &      0 \\
   0 &    0 &    0 &      0 &     0 &      0 &  -5/16 &     0 &  -1/16 &      0 & -1/16 &    %%@
0 &  -1/16 \\
   0 &  1/8 &    0 &    3/8 &     0 &    5/8 &      0 &     0 &      0 &      0 &     0 &    %%@
0 &      0 \\
   0 &    0 &    0 &      0 &     0 &      0 & 105/32 &     0 &  17/32 &      0 &  5/32 &    %%@
0 & -15/32 \\
   0 & 5/16 &    0 &   1/16 &     0 & -15/16 &      0 &     0 &      0 &      0 &     0 &    %%@
0 &      0 \\
   0 &    0 &    0 &      0 &     0 &      0 & -63/16 &     0 &  -3/16 &      0 & 13/16 &    %%@
0 &  -3/16 \\
   0 & 5/16 &    0 & -33/16 &     0 &  33/16 &      0 &     0 &      0 &      0 &     0 &    %%@
0 &      0 \\
   0 &    0 &    0 &      0 &     0 &      0 & 231/32 &     0 & -33/32 &      0 & 11/32 &    %%@
0 &  -1/32 \\
\end{array} \right)
\end{equation}
}
%\end{widetext}

%%% ----------------------------------------------------------------------

\subsubsection{Reverse Rotation Matrices} \label{ap-mcph2cfMat}

These matrices rotate a set of crystal field parameters (or Stevens operators) by %%@
$\Theta_1$=90$^o$ and $\Phi_1$=90$^o$ and again by $\Theta_2$=90$^o$ and $\Phi_2$=90$^o$.

\begin{equation} \label{eq:R2}
\V{S^{-1}}_2(\pi/2,\pi/2) = \left(
\begin{array}{ccccc}
 0 & -1/2 &    0 &    0 &    0 \\
 0 &    0 &    0 &    1 &    0 \\
 0 &    0 & -1/2 &    0 &  1/2 \\
 2 &    0 &    0 &    0 &    0 \\
 0 &    0 & -3/2 &    0 & -1/2 \\
\end{array} \right)
\end{equation}

%\begin{widetext}
\begin{equation} \label{eq:R4}
\V{S^{-1}}_4(\pi/2,\pi/2) = \left(
\begin{array}{ccccccccc}
 0 &  1/8 &    0 &  7/8 &    0 &    0 &    0 &    0 &    0 \\
 0 &    0 &    0 &    0 &    0 & -7/4 &    0 & -3/4 &    0 \\
 0 & -1/4 &    0 &  1/4 &    0 &    0 &    0 &    0 &    0 \\
 0 &    0 &    0 &    0 &    0 & -3/4 &    0 &  1/4 &    0 \\
 0 &    0 &    0 &    0 &  3/8 &    0 & -1/8 &    0 &  1/8 \\
 1 &    0 & -1/2 &    0 &    0 &    0 &    0 &    0 &    0 \\
 0 &    0 &    0 &    0 &  5/2 &    0 & -1/2 &    0 & -1/2 \\
-1 &    0 & -7/2 &    0 &    0 &    0 &    0 &    0 &    0 \\
 0 &    0 &    0 &    0 & 35/8 &    0 &  7/8 &    0 &  1/8 \\
\end{array} \right)
\end{equation}

{\scriptsize
\begin{equation} \label{eq:R6}
\V{S^{-1}}_6(\pi/2,\pi/2) = \left(
\begin{array}{ccccccccccccc}
   0 &  -1/32 &      0 & -11/32 &      0 & -33/32 &      0 &      0 &      0 &      0 &      %%@
0 &      0 &      0 \\
   0 &      0 &      0 &      0 &      0 &      0 &      0 &  33/16 &      0 &  33/16 &      %%@
0 &   5/16 &      0 \\
   0 &    1/8 &      0 &    3/8 &      0 &   -3/8 &      0 &      0 &      0 &      0 &      %%@
0 &      0 &      0 \\
   0 &      0 &      0 &      0 &      0 &      0 &      0 &  15/16 &      0 &  -1/16 &      %%@
0 &  -5/16 &      0 \\
   0 &  -5/32 &      0 &   9/32 &      0 &  -5/32 &      0 &      0 &      0 &      0 &      %%@
0 &      0 &      0 \\
   0 &      0 &      0 &      0 &      0 &      0 &      0 &    5/8 &      0 &   -3/8 &      %%@
0 &    1/8 &      0 \\
   0 &      0 &      0 &      0 &      0 &      0 &  -5/16 &      0 &   1/16 &      0 &  %%@
-1/16 &      0 &   1/16 \\
 3/4 &      0 &   -1/2 &      0 &    1/4 &      0 &      0 &      0 &      0 &      0 &      %%@
0 &      0 &      0 \\
   0 &      0 &      0 &      0 &      0 &      0 &-105/32 &      0 &  17/32 &      0 &  %%@
-5/32 &      0 & -15/32 \\
-5/8 &      0 &   -5/4 &      0 &    9/8 &      0 &      0 &      0 &      0 &      0 &      %%@
0 &      0 &      0 \\
   0 &      0 &      0 &      0 &      0 &      0 & -63/16 &      0 &   3/16 &      0 &  %%@
13/16 &      0 &   3/16 \\
 3/8 &      0 &   11/4 &      0 &   33/8 &      0 &      0 &      0 &      0 &      0 &      %%@
0 &      0 &      0 \\
   0 &      0 &      0 &      0 &      0 &      0 &-231/32 &      0 & -33/32 &      0 & %%@
-11/32 &      0 &  -1/32 \\
\end{array} \right)
\end{equation}
}



