\section{Definition of sipf file format}

\begin{verbatim}

#!MODULE=cfield
#!<--mcphase.sipf-->
#
#----------------------------------------------------------------------------
# obligatory: 
# the first line denotes which single ion module is used, it starts always with a modulename, 
# which is either the name of an internal single ion module or the filename of an external
# (possibly user written) module, which is loaded at runtime
#
# the second line identifies the file as a sipf file  #!<--mcphase.sipf-->
#
#
# in what follows all comment lines start with #
# variables are declared in uncommented lines only by the = sign
#----------------------------------------------------------------------------
#
#----------------------------------------------------------------------------
# optional for external modules a list of parameters
#----------------------------------------------------------------------------
MODPAR1=13
MODPAR2=1
MODPAR3=4.2

#***************************************************************
# Single Ion Parameter File for internal Module Kramer for  
# mcphas version 3.0
# - program to calculate static magnetic properties
# reference: M. Rotter JMMM 272-276 (2004) 481
# mcdisp version 3.0
# - program to calculate the dispersion of magnetic excitations
# reference: M. Rotter et al. J. Appl. Phys. A74 (2002) 5751
# mcdiff version 3.0
# - program to calculate neutron and magnetic xray diffraction
# reference: M. Rotter and A. Boothroyd PRB 79 (2009) 140405R
#***************************************************************
# this is a crystal field ground state doublet
# module, parameters are the following 3 matrix
# elements
#
# A=|<+-|Ja|-+>| B=|<+-|Jb|-+>| C=|<+-|Jc|+->| are obligatory
# parameters for module kramer, not read by other modules
 A=  2.000000 
 B=  2.543750 
 C=  1.600000

#***************************************************************
# Single Ion Parameter File for internal Module Brillouin for
# mcphas version 3.0
# - program to calculate static magnetic properties
# reference: M. Rotter JMMM 272-276 (2004) 481
# mcdisp version 3.0
# - program to calculate the dispersion of magnetic excitations
# reference: M. Rotter et al. J. Appl. Phys. A74 (2002) 5751
# mcdiff version 3.0
# - program to calculate neutron and magnetic xray diffraction
# reference: M. Rotter and A. Boothroyd PRB 79 (2009) 140405R
#****************************************************************
#
#
# single ion parameterized by Brillouin function
# BJ(x) with angular momentum number J=S,
# no crystal field
#
#  J   angular momentum quantum number is
# and obligatory parameter for module Brillouin, not read
# by other modules
J=3.5

#***************************************************************
# Single Ion Parameter File for internal Module Cfield for  
# mcphas version 3.0
# - program to calculate static magnetic properties
# reference: M. Rotter JMMM 272-276 (2004) 481
# mcdisp version 3.0
# - program to calculate the dispersion of magnetic excitations
# reference: M. Rotter et al. J. Appl. Phys. A74 (2002) 5751
# mcdiff version 3.0
# - program to calculate neutron and magnetic xray diffraction
# reference: M. Rotter and A. Boothroyd PRB 79 (2009) 140405R
#***************************************************************
#
#
# crystal field paramerized in Stevens formalism
#
#----------------------------------------------------------------------------
# IONTYPE is an obligatory parameter for internal module cfield -
#                          optional parameter for external module ic1ion
#                          not needed for other internal modules 
IONTYPE=Ce3+
#---------------------------------------------------------------------------

#--------------------------------------------------------------------------
# Crystal Field parameters in Stevens Notation (coordinate system yzx||abc in 
# in module cfield, xyz||abc in external module ic1ion)
# used by programs mcphas, mcdisp, mcdiff
#--------------------------------------------------------------------------
# corresponding rotated CF suited for mcphas yzx||abc
# the units statement is recognised only in external module ic1ion and defines the 
# units used from this point onwards
units=meV
# crystal field parameters Blm with l=2,4,6 m=-l,-l+1,...,+l
# Blm with negative l are denoted e.g. as B21S
B20=-0.022
B22=0.066
B40=0.0655
B42=0.098
B44=0.2295

#--------------------------------------------------------------------------
# external module ic1ion also recognises crystal field parameters in Wybourne notation
# used by programs mcphas, mcdisp, mcdiff
#--------------------------------------------------------------------------
# the units statement is recognised only in external module ic1ion and defines the 
# units used from this point onwards
units=meV
# crystal field parameters Llm with l=2,4,6 m=-l,-l+1,...,+l
# Llm with negative l are denoted e.g. as L21S
L20=-0.022
L22=0.066
L40=0.0655
L42=0.098
L44=0.2295

#-------------------------------------------------
# Stevens Factors (optional) used in module cfield
# used by programs mcphas, mcdisp, mcdiff
#-------------------------------------------------
ALPHA=-0.0571429
BETA=0.00634921
GAMMA=0

#---------------------------------------------------------
# (for use in programs pointc, chrgplt, charges) 
# radial wave function parameters, for transition metal ions the the values are tabulated in
# Clementi & Roetti Atomic data and nuclear data tables 14 (1974) 177-478, the radial wave
# function is expanded as R(r)=sum_p Cp r^(Np-1) . exp(-XIp r) . (2 XIp)^(Np+0.5) / sqrt(2Np!)
# Co2+ is isoelectronic to Fe+, looking at page  422 of Clemente & Roetti we see:
#---------------------------------------------------------
N1=3 XI1=4.95296 C1=0.36301 
N2=3 XI2=12.2963 C2=0.02707 
N3=3 XI3=7.03565 C3=0.14777
N4=3 XI4=2.74850 C4=0.49771 
N5=3 XI5=1.69027 C5=0.11388 

#---------------------------------------------------------
# Radial Matrix Elements (e.g. Abragam Bleaney 1971 p 399)
# for program pointc alternatively to radial wavefunction parameters,
#  the radial matrix elements may be given:
#---------------------------------------------------------
#<r^2> in units of a0^2 a0=0.5292 Angstroem
R2=1.24653
#<r^4> in units of a0^4 a0=0.5292 Angstroem
R4=3.67281
#<r^6> in units of a0^6 a0=0.5292 Angstroem
R6=21.0652

#------------------------------------------
# Lande factor gJ
# used by programs mcphas, mcdisp, mcdiff
#------------------------------------------
GJ=0.857143

#-------------------------------------------------------
# Neutron Scattering Length (10^-12 cm) (can be complex)
# used by program mcdiff
# a table for all elements is available at 
# http://www.ncnr.nist.gov/resources/n-lengths/list.html
#-------------------------------------------------------
SCATTERINGLENGTHREAL=0.484
SCATTERINGLENGTHIMAG=0
#  ... note: - if an occupancy other than 1.0 is needed, just reduce 
#              the scattering length linear accordingly

#-------------------------------------------------------
# Debye-Waller Factor: sqr(Intensity)~|sf|~EXP(-2 * DWF *s*s)=EXP (-W)
#                      with s=sin(theta)/lambda=Q/4pi
# relation to other notations: 2*DWF=Biso=8 pi^2 <u^2>
# unit of DWF is [A^2]
# used by program mcdiff, mcdisp, mcphas
#-------------------------------------------------------
DWF=0.00211

#--------------------------------------------------------------------------------------
# Neutron Magnetic Form Factor coefficients - thanks to J Brown
#   d = 2*pi/Q      
#   s = 1/2/d = Q/4/pi   
#   sin(theta) = lambda * s
#   s2= s*s = Q*Q/16/pi/pi
#
#   <j0(Q)>=    FFj0A*EXP(-FFj0a*s2) + FFj0B*EXP(-FFj0b*s2) + FFj0C*EXP(-FFj0c*s2) + FFj0D
#   <j2(Q)>=s2*(FFj2A*EXP(-FFj2a*s2) + FFj2B*EXP(-FFj2b*s2) + FFj2C*EXP(-FFj2c*s2) + FFj2D
#   <j4(Q)>=s2*(FFj4A*EXP(-FFj4a*s2) + FFj4B*EXP(-FFj4b*s2) + FFj4C*EXP(-FFj4c*s2) + FFj4D
#   <j6(Q)>=s2*(FFj6A*EXP(-FFj6a*s2) + FFj6B*EXP(-FFj6b*s2) + FFj6C*EXP(-FFj6c*s2) + FFj6D
#
#   Dipole Approximation for Neutron Magnetic Formfactor:
#        -Spin Form Factor       FS(Q)=<j0(Q)>
#        -Angular Form Factor    FL(Q)=<j0(Q)>+<j2(Q)>
#        -Rare Earth Form Factor F(Q) =<j0(Q)>+<j2(Q)>*(2/gJ-1)
# used by programs mcphas, mcdiff, mcdisp
# a table for all magnetic ions is available in file ffacts.tex
#--------------------------------------------------------------------------------------
FFj0A=+0.2953 FFj0a=+17.6846 FFj0B=+0.2923 FFj0b=+6.7329 FFj0C=+0.4313 FFj0c=+5.3827 FFj0D=-0.0194
FFj2A=+0.9809 FFj2a=+18.0630 FFj2B=+1.8413 FFj2b=+7.7688 FFj2C=+0.9905 FFj2c=+2.8452 FFj2D=+0.0120
FFj4A=-0.6468 FFj4a=+10.5331 FFj4B=+0.4052 FFj4b=+5.6243 FFj4C=+0.3412 FFj4c=+1.5346 FFj4D=+0.0080
FFj6A=-0.1212 FFj6a=+7.9940 FFj6B=-0.0639 FFj6b=+4.0244 FFj6C=+0.1519 FFj6c=+1.0957 FFj6D=+0.0078


#----------------------------------------------------------------------
# coefficients of Z(K') according to Lovesey (Neutron Scattering) vol.2
# chapter 11.6.1 page 233: Z(K)= ZKcK-1 * <jK-1(Q)> + ZKcK+1 * <jK+1(Q)>
#  ... these coefficients are needed to go beyond dipolar approx.
#      for the neutron magnetic formfactor in rare earth ions
#      - a table for all rare earth 3+ ions is available in file zk.tex
# optional parameters for module cfield, used in programs mcdiff and mcdisp
#----------------------------------------------------------------------
Z1c0=+1.07142857  Z1c2=+1.71428571
		  Z3c2=+0.09583148  Z3c4=+0.31943828
				    Z5c4=+0.00360750  Z5c6=+0.04329004
						      Z7c6=+0.00000000


#------------------------------------------------------------------------------------------------------
# configuration and electron electron interaction parameters to be used if not
# IONTYPE is given in external module ic1ion
#------------------------------------------------------------------------------------------------------
 conf=d6
 F0=0
 F2=88025
 F4=64512
 xi=514.5

#----------------------------------------------------------------------
# optional parameters for external module ic1ion for calculation of 
# magnetisation (triggered by the keyword clacmag)
#----------------------------------------------------------------------
# magnetisation calculation can be triggered by the following parameters
calcmag
# default magnetisation output units is Bohr magneton per ion

# if the following is uncommented, ic1ion will output magnetisation in emu/mol
#emu

# if the following is uncommented, ic1ion will output magnetisation in Am^2/mol
#simag
 xT   = 1
 xHa  = 0
 xHb  = 0
 xHc  = 0
 xmin = 1
 xstep= 1
 xmax = 300
 yT   = 0
 yHa  = 0
 yHb  = 0
 yHc  = 1
 ymin = 1
 ystep= 1
 ymax = 1


# keyword arnoldi: optional parameter for module ic1ion
# if present ic1ion will use 
# a routine to calculate only a few of the lowest
# eigenvalues in mcalc() using an arnoldi iterative method in ARPACK
# it makes calculation with f-configurations with many
# electrons (f5-f9) more practicable... It usually gets the correct
# eigenvalue/vector, however sometime for some reason (defective random
# number generator?) it gives completely erroneous numbers. This happens
# perhaps between once in a 100 times, and occurs randomly... I have no
# idea why it happens, and it doesn't occur systematically enough for me
# to track down the problem...
 arnoldi

# keyword partial: optional parameter for module ic1ion
# there is also another method to find only a few of the lowest
# eigenvalue, which is more reliable, and this uses the relatively
# robust representation algorithm in Lapack and can be accessibly by including the
# keyword: "partial" in the parameter file. This takes about 2/3 the
# time of the full computation, whereas the arnoldi routine takes about
# 1/5 the time...
# if the next line is uncommented ic1ion will use the 'partial' RRR algorithm 
 partial


#---------------------------------------------------------
#pointcharges charge[|e|]  x[A] y[A] z[A]
# output of program pointc, used in chrgplt
#---------------------------------------------------------
pointcharge=  0.3            3    4    2


\end{verbatim}
